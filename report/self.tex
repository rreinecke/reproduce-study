\textbf{S111}

What software licences do you use? Results: [None 'none' 'hfvjhgjh' 'GPL' 'MIT' 'Creative Commons'
 'CECILL-C Licence (a French equivalent to the L-GPL licence)'
 'Creative Commons licenses' 'None' 'MIT, LGPL' 'MIT / GPL3' 'GPLv3'
 'Creative Common' 'GNU' 'Apache License v2' 'GPL MIT' 'GNU GPL v3'
 'MIT license' 'CC BY-SA' 'Apache License' 'BSD' 'mit' 'GPL, PD, WTFPL'
 'MATLAB' 'GNU General Public License (GPL)' 'Gpl v3' 'GPL, MIT' 'CC'
 'mit, ah hoc' '???' 'GPL, MIT, Apache'
 'GNU General Public License (GPL), BSD 3-Clause' 'Apache 2.0'
 'US Government work - Public Domain' 'NONE' 'BSD 3-Clause' 'GPL3'
 'Apache' 'FreeBSD' 'SWMM' 'BSD, GNU' 'GPL or CC-4.0 non-commercial'
 'BSD 3-Clause, MIT' 'GNU 3, MIT' 'MIT, GPL' 'GNU GPL' 'educational'
 'MIT, GNUv3, Creative Commons' 'GNU General Public License'
 'MIT, GPL, or commercial (depending on use)' 'MatLab' 'BSD3, GPL'
 'GPL v3' 'NOne' 'BSD-2-Clause License' 'GNU LGPL-2.1 License' 'MIT, BSD'] 

\textbf{S112}

Which of the licences are you familar with?

Other licences mentioned: [None 'CC0, Unlicense, WTFPL' 'LGPL' 'PD, WTFPL' 'None' 'Creative Commons'
 'ArcGIS, Django' "ESRI's ArcGIS" 'MatLab' 'MIT']

Alternative answers. I dont know them at all: 49, I have heard of them but have no idea what they stand for: 69
Number of times licences were recongnized: 

 Apache License 33

 BSD 3-Clause 24

 FreeBSD 17

 GNU General Public License (GPL) 119

 MIT license 72

 Mozilla Public License 23

 Common Development and Distribution License 34

 Other 9

\textbf{S101}

What kind of programming languages do you mainly use? 

Other languages mentioned: ['Matlab', 'MATLAB', 'NCL', 'GAMS', 'matlab', 'ECMAScript', 'bash', 'Java', 'MatLab', 'Linux bash', 'shell', 'MPI', 'Stella', 'OpenFOAM', 'Delphi', 'PERL, MatLab, Pascal', 'Bash', 'Legacy (meta) languages in MS Office and ESRI: VBcode, AML (1981-1999), model builder. TBX ', 'Matlab and if possible I try to provided also compiled versione for complete reproducibility', 'whatever the students want to learn and /or makes sense for a particular project', 'Basic', 'Matlab, Julia', 'Octave/Matlab', 'Julia', 'Matlab, IDL']

Number of times languages were selected: 

 Fortran 51

 C/C++ 64

 Python 146

 R 118

 Others 72


None of the above 13

\textbf{S104}

What methods are you applying?

Others mentioned: [None 'Procedural programming' 'UML' 'Continuous Integration'
 'Validation by comparison of numercial results with analytical solution or code-to-code intercomparison'
 'None' 'Yelling at machine'
 'git reviews with other researchers on the same project'
 'continuous-integration' 'Maybe?' 'team code review'
 'Open-source development (issues / pull-requests on GitHub, GitLab, etc.)']

Number of times methods were selected: 

 Pair programming 14

 Scrum 20

 Test-driven development 43

 Object-oriented programming 108

 Functional programming 67

 Software Design Patterns (such as Abstract factory or Decorator or similar) 19

 Automated Testing 47

 Other 12

None of the above: 104

\textbf{S105}

What tools are you using?

Others mentioned: [None 'Pycharm' 'Automatic formatting; issue tracker'
 'ReadTheDocs/TravisCi/PyCharm/GitHub/' 'Doxygen ' 'Code quality Checkers'
 'Markdown' 'Continuous Integration tools: TravisCI, CircleCI'
 'Cloud drive storage such as Box to maintain an archive of previous versions'
 'No'
 'jupyter hubs to make sure all members of the team work on the same distribution'
 'Autoformatters, Pre-commit hooks, Test Coverage, ' 'Jupyter notebook'
 'auto-formatters, linters, test frameworks, continuous integration tools'
 'Continuous integration']

Number of times tools were selected: 

 Version control such as Git, SVC or similar 156

 IDEs such as Eclipse or similar 85

 Automated documentation such as Docstrings or similar 52

 Other tools 14

None of the above 84

\textbf{S203}

What keeps you from publishing as Open Source?

Other reasons mentioned: [None "I don't write code"
 "I publish my code in out institutions library network, which can be freely accessed, but I'm not sure if this qualifies per se open source."
 'The model I am developing is not mine and the owners do not want to make it open-source. '
 'Code is unique for a certain data set. '
 'code not ready - still in development phase'
 'I try to publish most, but often time is lacking to bring it into a shape that it can be used by anyone (all bugs checked, properly documented)'
 'if possible in terms of time, codes have been published'
 'Too shy, worried about code quality' "Haven't published any yet"
 'I don?t know if it would be useful'
 "Haven't come to completion of PhD projects yet, but I plant to make everything open source"
 'not good enough to publish' 'Supervisor opposition '
 'The exceutable files are available on the web as a package that is regularly updated'
 'most of software has been developped during non-paid time and/or for personal consulting purposes. Some concern on giving away "for free" all of this immense invested non-paid time. Further, my code is relatively simple.'
 'LPJ-GUESS model that we frequenctly use is not openly licensed by the main development group in Lund.'
 'sensitive data/privacy and legal issues'
 'I have not spent a lot of time thinking about the subject, but I do publish my code on github'
 'Priority. I publish open data... I publish maps... I include scripts/code in some datasets... For others, it is so legacy format, that I archive it privately. '
 'not sure how and what are the standards' 'No specific reason']


Hurdles that were selected: 

 Licence (too complex to understand which to pick or restricted by university/institution) 42

 Complexity (code too complex, not enough documentation) 40

 Competition (fear to lose lead on other groups) 24

 Technical ressources (e.g. storage) 21

 Time 30

 Other 38

 Funding 84

 Staff resources 21

I publish all my code as open source: 74

I don't want to: 5

Does not apply to me: 41

\textbf{S106}

Where did you learn to programm?


Trainings that were selected: 

 Computer Science degree 7

 Courses during undergraduate studies (e.g., BA / Bsc) 102

 Courses during postgraduate studies (e.g., MA / MSc / PhD) 95

 Workshops 59

 Online-courses 62

 Self-taught /autodidact 212

I'm not able to write my own code: 9

